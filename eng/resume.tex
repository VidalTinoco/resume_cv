%!TEX TS-program = xelatex
%!TEX encoding = UTF-8 Unicode
% Awesome CV LaTeX Template for CV/Resume
%
% This template has been downloaded from:
% https://github.com/posquit0/Awesome-CV
%
% Author:
% Claud D. Park <posquit0.bj@gmail.com>
% http://www.posquit0.com
%
%
% Adapted to be an Rmarkdown template by Mitchell O'Hara-Wild
% 23 November 2018
%
% Template license:
% CC BY-SA 4.0 (https://creativecommons.org/licenses/by-sa/4.0/)
%
%-------------------------------------------------------------------------------
% CONFIGURATIONS
%-------------------------------------------------------------------------------
% A4 paper size by default, use 'letterpaper' for US letter
\documentclass[11pt,a4paper,]{awesome-cv}

% Configure page margins with geometry
\usepackage{geometry}
\geometry{left=1.4cm, top=.8cm, right=1.4cm, bottom=1.8cm, footskip=.5cm}


% Specify the location of the included fonts
\fontdir[fonts/]

% Color for highlights
% Awesome Colors: awesome-emerald, awesome-skyblue, awesome-red, awesome-pink, awesome-orange
%                 awesome-nephritis, awesome-concrete, awesome-darknight

\definecolor{awesome}{HTML}{414141}

% Colors for text
% Uncomment if you would like to specify your own color
% \definecolor{darktext}{HTML}{414141}
% \definecolor{text}{HTML}{333333}
% \definecolor{graytext}{HTML}{5D5D5D}
% \definecolor{lighttext}{HTML}{999999}

% Set false if you don't want to highlight section with awesome color
\setbool{acvSectionColorHighlight}{true}

% If you would like to change the social information separator from a pipe (|) to something else
\renewcommand{\acvHeaderSocialSep}{\quad\textbar\quad}

\def\endfirstpage{\newpage}

%-------------------------------------------------------------------------------
%	PERSONAL INFORMATION
%	Comment any of the lines below if they are not required
%-------------------------------------------------------------------------------
% Available options: circle|rectangle,edge/noedge,left/right

\name{}{Vidal Mendoza Tinoco}

\address{Political Scientist -- Coyoacán, Mexico City, Mexico}

\email{\href{mailto:vidalhum0@gmail.com}{\nolinkurl{vidalhum0@gmail.com}}}
\github{VidalTinoco}

% \gitlab{gitlab-id}
% \stackoverflow{SO-id}{SO-name}
% \skype{skype-id}
% \reddit{reddit-id}


\usepackage{booktabs}

\providecommand{\tightlist}{%
	\setlength{\itemsep}{0pt}\setlength{\parskip}{0pt}}

%------------------------------------------------------------------------------



% Pandoc CSL macros
\newlength{\cslhangindent}
\setlength{\cslhangindent}{1.5em}
\newlength{\csllabelwidth}
\setlength{\csllabelwidth}{2em}
\newenvironment{CSLReferences}[2] % #1 hanging-ident, #2 entry spacing
 {% don't indent paragraphs
  \setlength{\parindent}{0pt}
  % turn on hanging indent if param 1 is 1
  \ifodd #1 \everypar{\setlength{\hangindent}{\cslhangindent}}\ignorespaces\fi
  % set entry spacing
  \ifnum #2 > 0
  \setlength{\parskip}{#2\baselineskip}
  \fi
 }%
 {}
\usepackage{calc}
\newcommand{\CSLBlock}[1]{#1\hfill\break}
\newcommand{\CSLLeftMargin}[1]{\parbox[t]{\csllabelwidth}{\honortitlestyle{#1}}}
\newcommand{\CSLRightInline}[1]{\parbox[t]{\linewidth - \csllabelwidth}{\honordatestyle{#1}}}
\newcommand{\CSLIndent}[1]{\hspace{\cslhangindent}#1}

\begin{document}

% Print the header with above personal informations
% Give optional argument to change alignment(C: center, L: left, R: right)
\makecvheader

% Print the footer with 3 arguments(<left>, <center>, <right>)
% Leave any of these blank if they are not needed
% 2019-02-14 Chris Umphlett - add flexibility to the document name in footer, rather than have it be static Curriculum Vitae
\makecvfooter
  {November 2023}
    { Vidal Mendoza Tinoco~~~·~~~Resume}
  {\thepage}


%-------------------------------------------------------------------------------
%	CV/RESUME CONTENT
%	Each section is imported separately, open each file in turn to modify content
%------------------------------------------------------------------------------



\hypertarget{working-experience}{%
\section{Working Experience}\label{working-experience}}

\begin{cventries}
    \cventry{Mexico’s  Federal Judicial Power Election Tribunal (TEPJF in Spanish)}{Jr Data Scientist}{Coyoacán, Mexico City}{September, 2023 – Today}{\begin{cvitems}
\item In charge of creating a ShinyApp as well as R packages to ease data extraction using SQL queries  from the General Agreements Ministry’s Information System (SISGA in Spanish), in order to respond to requests for information made through the National Transparency Portal.
\item Collaboration in the creation of machine learning models (Convolutional Neural Networks) using Keras and TensorFlow, to identify handwritten data in the official records of the Mexican National Electoral Institute. This way, making the process of challenges and resolutions in the 2024 election process more efficient.
\end{cvitems}}
    \cventry{State of Michoacán’s Finances Ministry – Department Head}{Data Scientist}{Morelia, Michoacán}{November 2021 - September 2023}{\begin{cvitems}
\item Based on tax information and using supervised machine learning techniques (lasso and ridge regression), I built a couple of models to predict trends in tax collection (forecasting) as well as estimating tax multipliers (such as fiscal stress). This way, we monitored the state’s government’s ability to meet its financial obligations, resulting in the restructuring of public debt.
\item Debt restructuring saved 540 million Mexican pesos a year, which were invested in several social programs. Public spending was directed to various sectors of the population to comply with the state government's plan using unsupervised machine learning methods (K‑means clustering).
\item I was in charge of building several databases by extracting text from physical documents. I did this using a supervised machine learning model that classifies the text according to its structure and content.
\end{cvitems}}
    \cventry{State of Michoacán’s Finances Ministry – Department Head}{Data Analysis Teacher}{Morelia, Michoacán}{May 2023 - July 2023}{\begin{cvitems}
\item I designed and taught a data analysis course using R to the staff of the State of Michoacán’s Finances Ministry. Introduction to R programming language, manipulating data using Dplyr, data visualization using Ggplot2 and Plotly, are among the topics I taught, in addition to creating reports using RMarkdown.
\end{cvitems}}
    \cventry{City Council of Erongarícuaro, in the State of Michoacán – External Consultant}{Public Policy Consultant}{Erongarícuaro, Michoacán (Hybrid)}{December 2021 - June 2022}{\begin{cvitems}
\item Data collection using surveys and discussion panels in the municipality of Erongarícuaro in the State of Michoacán to identify development issues at municipal level.
\item The behavior of several social issues such as violence and the availability of drinking water during the year were estimated using time series regressions that included seasonal dichotomous variables.
\item The municipal development plan was created based on the suggested analyses, where public policies are structured to prevent and relieve obstacles for the municipality’s social welfare.
\end{cvitems}}
    \cventry{Instituto Tecnológico Autónomo de México, ITAM (Mexican Autonomous Technological Institute)}{Research Assistant}{Álvaro Obregón, Ciudad de México}{August 2019 - May 2020}{\begin{cvitems}
\item Research assistant in the development of research projects by Dr. Eric Magar (ITAM).
\item I created and kept several databases updated using Git and GitHub. Such databases were related to research in the subject of public election. One of them was about the change in electoral preferences in Mexico.
\end{cvitems}}
\end{cventries}

\hypertarget{schooling}{%
\section{Schooling}\label{schooling}}

\begin{cventries}
    \cventry{Bachelor's Degree in Political Science}{Instituto Tecnológico Autónomo de México, ITAM (Mexican Autonomous Technological Institute)}{Álvaro Obregón, Ciudad de México}{August 2017 - July 2021}{\begin{cvitems}
\item Dissertation: Green Gold Splashes. Effects of the Avocado Boom on Income Inequality in the State of Michoacán.
\item Relevant courses: econometrics, causal inference, statistical inference, probability, algebra (including matrix algebra), differential and integral calculus.
\end{cvitems}}
    \cventry{Data and Policy Summer Scholar Program (DPSS)}{Harris School of Public Policy – University of Chicago}{Chicago, Illinois}{Summer 2021}{\begin{cvitems}
\item Final Research Project:  Quantitative Analysis of the Protests That Were Causes by the Murder of George Floyd in the United States.
\item Relevant Courses: Data analysis for public policies.
\end{cvitems}}
\end{cventries}

\pagebreak

\hypertarget{projects}{%
\section{Projects}\label{projects}}

\begin{cventries}
    \cventry{Instituto Tecnológico Autónomo de México, ITAM (Mexican Autonomous Technological Institute) - Supervised by Dra. Antonella Bandiera}{Research - Green Gold Splashes}{Remote}{2022-2023}{\begin{cvitems}
\item My research is about the way that the shock on the demanding of a product such as avocado affects income inequality in the areas where it is grown. To do this, I used data from formal income in the state of Michoacán to calculate the Gini coefficient per municipality from 2003 to 2020. I compared these data through a difference in difference design (Diff in Diff), using Event Study Models and Two Way Fixed Effects.
\end{cvitems}}
\end{cventries}

\hypertarget{skills}{%
\section{Skills}\label{skills}}

\begin{cventries}
    \cventry{Programming languages}{Technical skills}{}{}{\begin{cvitems}
\item R, RMarkdown, Python, SQL, Git/GitHub and Latex.
\end{cvitems}}
    \cventry{Packages}{}{}{}{\begin{cvitems}
\item Tidyverse, Dplyr, Ggplot2, Plotly, Caret, Pandas, Numpy, Keras and TensorFlow.
\end{cvitems}}
    \cventry{English}{Languages}{}{}{\begin{cvitems}
\item Fluent | IELTS: 7.5
\end{cvitems}}
    \cventry{Spanish}{}{}{}{\begin{cvitems}
\item Native speaker
\end{cvitems}}
\end{cventries}



\end{document}
