%!TEX TS-program = xelatex
%!TEX encoding = UTF-8 Unicode
% Awesome CV LaTeX Template for CV/Resume
%
% This template has been downloaded from:
% https://github.com/posquit0/Awesome-CV
%
% Author:
% Claud D. Park <posquit0.bj@gmail.com>
% http://www.posquit0.com
%
%
% Adapted to be an Rmarkdown template by Mitchell O'Hara-Wild
% 23 November 2018
%
% Template license:
% CC BY-SA 4.0 (https://creativecommons.org/licenses/by-sa/4.0/)
%
%-------------------------------------------------------------------------------
% CONFIGURATIONS
%-------------------------------------------------------------------------------
% A4 paper size by default, use 'letterpaper' for US letter
\documentclass[11pt,a4paper,]{awesome-cv}

% Configure page margins with geometry
\usepackage{geometry}
\geometry{left=1.4cm, top=.8cm, right=1.4cm, bottom=1.8cm, footskip=.5cm}


% Specify the location of the included fonts
\fontdir[fonts/]

% Color for highlights
% Awesome Colors: awesome-emerald, awesome-skyblue, awesome-red, awesome-pink, awesome-orange
%                 awesome-nephritis, awesome-concrete, awesome-darknight

\definecolor{awesome}{HTML}{414141}

% Colors for text
% Uncomment if you would like to specify your own color
% \definecolor{darktext}{HTML}{414141}
% \definecolor{text}{HTML}{333333}
% \definecolor{graytext}{HTML}{5D5D5D}
% \definecolor{lighttext}{HTML}{999999}

% Set false if you don't want to highlight section with awesome color
\setbool{acvSectionColorHighlight}{true}

% If you would like to change the social information separator from a pipe (|) to something else
\renewcommand{\acvHeaderSocialSep}{\quad\textbar\quad}

\def\endfirstpage{\newpage}

%-------------------------------------------------------------------------------
%	PERSONAL INFORMATION
%	Comment any of the lines below if they are not required
%-------------------------------------------------------------------------------
% Available options: circle|rectangle,edge/noedge,left/right

\name{}{Vidal Mendoza Tinoco}

\address{Ciencia Política - Coyoacán, Ciudad de México, México}

\email{\href{mailto:vidalhum0@gmail.com}{\nolinkurl{vidalhum0@gmail.com}}}
\github{VidalTinoco}

% \gitlab{gitlab-id}
% \stackoverflow{SO-id}{SO-name}
% \skype{skype-id}
% \reddit{reddit-id}


\usepackage{booktabs}

\providecommand{\tightlist}{%
	\setlength{\itemsep}{0pt}\setlength{\parskip}{0pt}}

%------------------------------------------------------------------------------



% Pandoc CSL macros
\newlength{\cslhangindent}
\setlength{\cslhangindent}{1.5em}
\newlength{\csllabelwidth}
\setlength{\csllabelwidth}{2em}
\newenvironment{CSLReferences}[2] % #1 hanging-ident, #2 entry spacing
 {% don't indent paragraphs
  \setlength{\parindent}{0pt}
  % turn on hanging indent if param 1 is 1
  \ifodd #1 \everypar{\setlength{\hangindent}{\cslhangindent}}\ignorespaces\fi
  % set entry spacing
  \ifnum #2 > 0
  \setlength{\parskip}{#2\baselineskip}
  \fi
 }%
 {}
\usepackage{calc}
\newcommand{\CSLBlock}[1]{#1\hfill\break}
\newcommand{\CSLLeftMargin}[1]{\parbox[t]{\csllabelwidth}{\honortitlestyle{#1}}}
\newcommand{\CSLRightInline}[1]{\parbox[t]{\linewidth - \csllabelwidth}{\honordatestyle{#1}}}
\newcommand{\CSLIndent}[1]{\hspace{\cslhangindent}#1}

\begin{document}

% Print the header with above personal informations
% Give optional argument to change alignment(C: center, L: left, R: right)
\makecvheader

% Print the footer with 3 arguments(<left>, <center>, <right>)
% Leave any of these blank if they are not needed
% 2019-02-14 Chris Umphlett - add flexibility to the document name in footer, rather than have it be static Curriculum Vitae
\makecvfooter
  {October 2023}
    { Vidal Mendoza Tinoco~~~·~~~Resume}
  {\thepage}


%-------------------------------------------------------------------------------
%	CV/RESUME CONTENT
%	Each section is imported separately, open each file in turn to modify content
%------------------------------------------------------------------------------



\hypertarget{experiencia}{%
\section{Experiencia}\label{experiencia}}

\begin{cventries}
    \cventry{Tribunal Electoral del Poder Judicial de la Federación}{Científico de Datos Jr.}{Coyoacán, Ciudad de México}{Septiembre 2023 - Hoy}{\begin{cvitems}
\item Extracción de datos mediante consultas SQL al Sistema de Información de la Secretaría General de Acuerdos (SISGA), para dar respuesta a solicitudes de información hechas vía el Portal Nacional de Transparencia.
\item Creación de paquetes de R para la sistematización y presentación de los datos extraídos del SISGA.
\item Colaboración en la creación de modelos de aprendizaje de máquina (Convolutional Neural Networks) con Keras y TensorFlow, para identificar datos escritos a mano en las actas del Instituto Nacional Electoral. Así, hacer más eficiente el proceso de impugnaciones y resoluciones en el proceso electoral del 2024.
\end{cvitems}}
    \cventry{Secretaría de Finanzas del Estado de Michoacán - Jefe de Departamento}{Científico de Datos}{Morelia, Michoacán}{Noviembre 2021 - Septiembre 2023}{\begin{cvitems}
\item Con base en datos fiscales, y usando técnicas supervisadas de aprendizaje de máquina (lasso y ridge regression) construí un par de modelos para predecir tendencias en la recaudación fiscal (forecasting) y la estimación de multiplicadores fiscales (como el estrés fiscal). Así, monitoreamos la capacidad del gobierno estatal para cumplir con sus obligaciones financieras, resultando en la reestructuración de la deuda pública.
\item La reestructuración de la deuda ocasionó un ahorro de 540 millones de pesos al año, destinados a la inversión en diferentes programas sociales. Mediante métodos de aprendizaje de máquina no supervisado (K-means clustering) se orientó el gasto público a diversos sectores poblacionales para cumplir con el plan de gobierno estatal.
\item Estuve encargado de construir diversas bases de datos a partir de la extracción de texto de documentos físicos. Ello se realizó mediante un modelo supervisado de aprendizaje de máquina que clasifica el texto de acuerdo a su estructura y contenido. 
\end{cvitems}}
    \cventry{Secretaría de Finanzas del Estado de Michoacán - Jefe de Departamento}{Profesor de Análisis de Datos}{Morelia, Michoacán}{Mayo 2023 - Julio 2023}{\begin{cvitems}
\item Diseño e impartición de un curso de análisis de datos con R para personal de la Secretaría de Finanzas del Estado de Michoacán. Entre los temas instruidos se encuentran la introducción al lenguaje R, manipulación de datos con Dplyr, visualización de datos con Ggplot2 y Plotly, además de la creación de reportes con RMarkdown.
\end{cvitems}}
    \cventry{Ayuntamiento de Erongarícuaro - Consultor Externo}{Consultor de Políticas Públicas}{Erongarícuaro, Michoacán (Híbrido)}{Diciembre 2021 - Junio 2022}{\begin{cvitems}
\item Recolección de datos mediante encuestas y mesas de trabajo en el municipio de Erongarícuaro, Michoacán para identificar problemas de desarrollo a nivel municipal.
\item Mediante regresiones de series de tiempo que incluyeron variables dicotómicas por temporadas se estimó el comportamiento de diversos problemas sociales como la violencia y la disposición de agua potable durante el año.
\item Con base en los análisis propuestos se creó el plan de desarrollo municipal, donde se estructuran políticas públicas para prevenir y mitigar obstáculos para el bienestar social del municipio.
\end{cvitems}}
    \cventry{Instituto Tecnológico Autónomo de México (ITAM)}{Asistente de Investigación}{Álvaro Obregón, Ciudad de México}{Agosto 2019 - Mayo 2020}{\begin{cvitems}
\item Asistencia en el desarrollo de proyectos de investigación con el Dr. Eric Magar (ITAM).
\item Usando Git y GitHub creé y mantuve actualizadas diversas bases de datos correspondientes a investigaciones en temas de elección pública. Una de ellas sobre el cambio en las preferencias electorales en México.
\end{cvitems}}
\end{cventries}

\hypertarget{educaciuxf3n}{%
\section{Educación}\label{educaciuxf3n}}

\begin{cventries}
    \cventry{Licenciatura en Ciencia Política}{Instituto Tecnológico Autónomo de México (ITAM)}{Álvaro Obregón, Ciudad de México}{Agosto 2017 - Julio 2021}{\begin{cvitems}
\item Tesina: Salpicaduras de Oro Verde. Efectos del boom aguacatero sobre la desigualdad salarial en Michoacán.
\item Cursos Relevantes: Econometria, inferencia causal, inferencia estadística, probabilidad, algebra (incluyendo algebra matricial), calculo diferencial e integral.
\end{cvitems}}
    \cventry{Programa Académico de Verano en Datos y Política Pública (DPSS por sus siglas en inglés)}{Harris School of Public Policy – University of Chicago}{Chicago, Illinois}{Verano 2021}{\begin{cvitems}
\item Proyecto de Investigación Final: Análisis Cuantitativo de las Protestas Originadas por el Asesinato de George Floyd en Estados Unidos.
\item Cursos Relevantes: Análisis de Datos para Políticas Públicas.
\end{cvitems}}
\end{cventries}

\pagebreak

\hypertarget{proyectos}{%
\section{Proyectos}\label{proyectos}}

\begin{cventries}
    \cventry{Instituto Tecnológico Autónomo de México - Supervisada por Dra. Antonella Bandiera}{Investigación - Salpicaduras de Oro Verde}{Remoto}{2022-2023}{\begin{cvitems}
\item Mi investigación trata sobre la manera en que el shock en la demanda de un producto como el aguacate afecta la desigualdad salarial en las zonas donde se cultiva. Para ello, utilicé datos de salarios formales de Michoacán para calcular el coeficiente de Gini por municipio desde 2003 hasta 2020. Comparé estos datos a través de un diseño de diferencias en diferencias (Diff in Diff), utilizando modelos de estudio de eventos (Event Study Model) y regresiones con efectos fijos (Two Way Fixed Effects).
\end{cvitems}}
\end{cventries}

\hypertarget{habilidades}{%
\section{Habilidades}\label{habilidades}}

\begin{cventries}
    \cventry{Lenguajes}{Habilidades Técnicas}{}{}{\begin{cvitems}
\item R, RMarkdown, Python, SQL y Latex.
\end{cvitems}}
    \cventry{Paquetería}{}{}{}{\begin{cvitems}
\item Tidyverse, Dplyr, Ggplot2, Plotly, Caret, Pandas, Numpy, Keras y TensorFlow.
\end{cvitems}}
    \cventry{Inglés}{Idiomas}{}{}{\begin{cvitems}
\item Fluido | IELTS: 7.5
\end{cvitems}}
    \cventry{Español}{}{}{}{\begin{cvitems}
\item Nativo
\end{cvitems}}
\end{cventries}



\end{document}
